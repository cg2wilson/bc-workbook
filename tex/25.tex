\documentclass[notes]{subfiles}
\begin{document}
	\addcontentsline{toc}{section}{2.5 - The Derivative, Algebraically}
	\refstepcounter{section}
	\fancyhead[RO,LE]{\bfseries  \large \nameref{cs25}} 
	\fancyhead[LO,RE]{\bfseries  \currentname}
	\fancyfoot[C]{{}}
	\fancyfoot[RO,LE]{\large \thepage}	%Footer on Right \thepage is pagenumber
	\fancyfoot[LO,RE]{\large Chapter 2.5}


\section*{The Derivative, Algebraically}\label{cs25}
	\begin{defn}[Derivative (Algebraic Definition)]
		Let $f(x)$ be a function defined on the open interval $(a,b)$, and $x\in (a,b)$.  Then, the derivative of $f$ at point $x$ is given by the formula
			\showto{ins}{\[\ds \lim_{h\to 0}\dfrac{f(x+h)-f(x)}{h}\]}\showto{st}{\[\vspace*{50pt}\]}
	\end{defn}

	\begin{question}
		Why is this definition the same as the one in \S2.4?
	\end{question}
		\vs{1}
		
	It is useful to remember a few things from algebra when doing these calculations: \showto{st}{\\}
	
	\begin{itemize}
		\item \showto{ins}{\fbox{$\sqrt{x + y}\neq \sqrt{x} + \sqrt{y}$}}\showto{st}{\blank{3}\\[10pt]}
		\item \showto{ins}{\fbox{$(a+b)^2 \neq a^2 + b^2$}}\showto{st}{\blank{3}\\[10pt]}
		\item \showto{ins}{\fbox{$\dfrac{a}{b+c}\neq \dfrac{a}{b} + \dfrac{a}{c}$}}\showto{st}{\blank{3}}
	\end{itemize}
		\vs{.25}
		
	When we algebraically find the derivative of a function, there is a four-step process which makes the algebra much simpler, and the derivative easier to find.  The steps are:\\
	\begin{enumerate}
		\item \showto{ins}{\fbox{Find and simplify $f(x+h)$}}\showto{st}{\blank{4}\\}
		\item \showto{ins}{\fbox{Find and simplify $f(x+h)-f(x)$}}\showto{st}{\blank{4} \\}
		\item \showto{ins}{\fbox{Find and simplify $\dfrac{f(x+h)-f(x)}{h}$}}\showto{st}{\blank{4}\\}
		\item \showto{ins}{\fbox{Compute $\ds \lim_{h\to 0} \dfrac{f(x+h)-f(x)}{h}$}}{\blank{4}}
	\end{enumerate}
		\vs{.25}
		
	This is demonstrated below.
		\vs{.25}
		\newpage
		
	\begin{ex}
		Algebraically find the derivative of the function $f(x) = x^2 $ using the four-step process. 
	\end{ex}
 		\newpage
 		
	\begin{ex} 
		Algebraically find the derivative of the function $f(x) = 5x -2$ using the four-step process.
	\end{ex}
		\newpage
		
	\begin{ex}
		The time it takes an average athlete to swim 100 meters freestyle at age $x$ years can be modeled as 
		\[t(x) = 0.181x^2 - 8.463x + 147.376\text{ seconds}\]
			\begin{enumerate}[(a)]
				\item Calculate the swim time at age 13 to the nearest second.
					\vs{.5}
				\item Use the algebraic method to develop a formula for the derivative of $t$ (ie, find $t'(x)$).	
					\vs{3}
				\item How quickly is the time to swim 100 meters freestyle changing for an average 13-year-old athlete?  Round to the nearest hundredth and interpret the result.
					\vs{.5}
				\item Compute the percent rate of change of swimmers' time at age 13, to the nearest tenth.
					\vs{.5}
			\end{enumerate}
	\end{ex}
		\newpage
		
	\begin{ex}
		Algebraically determine the derivative of $f(t) = \dfrac{1}{2}t^2 -\dfrac{1}{3}$, and evaluate $\dfrac{df}{dt}\bigg|_{t=1}$
	\end{ex}
		\newpage
		
	\begin{ex}
		Algebraically determine the derivative of $k(r) = r^2-2r^3$, and evaluate $k'(0)$
	\end{ex}
	
	\clearpage
\end{document}