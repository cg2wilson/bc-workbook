\documentclass{article}
\usepackage{fullpage}
\title{Some notes to take note of before using these notes}
\author{Thomas Lane}

\begin{document}
\maketitle
\begin{enumerate}
\item These notes were originally written by Cory Wilson and recently editted by Thomas Lane. The edits were mostly spacing and typo issues. I did prune some of the longer worksheets and condensed repeat problems. Some of the easier sections were condensed into a single worksheet. There are still some areas that could use improvement.

\item I've taught from these notes twice now, and they work pretty well. Don't follow the notes religiously; do what feels right and slow down/speed up/skip things/add examples based how your class responds to content. The notes are just a starting point, and for those of you who are IBL folks, these are IBL friendly. I personally do a mixture and do a majority of lecture but give them group activities from the notes almost every class meeting.

\item In each exam cycle, I often finished the notes with roughly 50-75 minutes of class time that I used for review (excluding the given review day in the schedule). That worked for my class in the previous semester. If you want to go through the notes slower, again, do what feels right!

\item The notes basically follow the textbook and some of the good problems are given as exercises in the notes. If students miss class, you can tell them to read the textbook and they can fill in the notes without much effort. 

\item Cory and I have not posted completed notes, and if you want to do that, please ask first. I personally use that as a motivation to come to class (which they need)! I print out the notes for them and put blank ones on canvas. Cory only puts them on canvas.

\item There are far too many exercises given in the notes to do in class. A good use of these is to assign some of these problems for homework and maybe put one of those on a quiz.

\item Some of the problems lack rounding guidelines. In business calculus 2, students are told to always round to two decimal places unless some other instruction/rule applies. You can use those problems to introduce this idea.

\item If you decide to use these notes, you might have some questions during the semester. Feel free to email me and ask! (thomas.r.lane@ou.edu)
\end{enumerate}

\end{document}