\documentclass[notes]{subfiles}
\begin{document}
	\addcontentsline{toc}{section}{1.6 - Models in Finance}
	\refstepcounter{section}
	\fancyhead[RO,LE]{\bfseries  \large \nameref{cs16}} 
	\fancyhead[LO,RE]{\bfseries \currentname}
	\fancyfoot[C]{{}}
	\fancyfoot[RO,LE]{\large \thepage}	%Footer on Right \thepage is pagenumber
	\fancyfoot[LO,RE]{\large Chapter 1.6}


\section*{Models in Finance}\label{cs16}
	\begin{defn}[Future Value/Present Value] The \textbf{future value} of an investment/loan at time $t$ is the sum of the prevent value and all accumulated interest; this is denoted $F$ or $FV$.  The \textbf{present value}, denoted $F(0) = P$ (principal) is the value ``today'', or at $t = 0$.
	\end{defn}
		\vspace{.1in}

	\subsection*{Simple Interest}
		\begin{defn}[Simple Interest] \textbf{Simple interest} is interest earned on \showto{ins}{\fbox{the present value only}}\showto{st}{\blank{2}} the rate (as a decimal) is \showto{st}{\\[15pt]} called the \showto{ins}{\fbox{annual percentage rate (APR)}}\showto{st}{\blank{2}}, or nominal rate.\end{defn}
		\vspace{.1in}

		We have two formulas for simple interest:
			\begin{center}
				\begin{tabular}{cc}
					$I(t)$		&= \fitb{$Prt$ dollars}{}\\[15pt]
					$F_s(t)$ 	&= \fitb{$P(1+rt)$ dollars}{}
				\end{tabular}
			\end{center}	
			\vspace{.1in}

		where $P$ is the \showto{ins}{\fbox{principal}}\showto{st}{\blank{1.5}}, $r$ is the \showto{ins}{\fbox{rate (as a decimal)}}\showto{st}{\blank{2}}, and $t$ is the \showto{ins}{\fbox{time (in years)}}\showto{st}{\\[15pt]\blank{2}}.
		
		\begin{ex} A family friend offers to loan you \$10,000 to cover your outlandishly high tuition this year.  She wants to earn 5.75\% interest on the loan.  
			\begin{enumerate}[(a)]
				\item If you pay the loan back in 1 year, how much interest does the friend make?
					\vs{1}

				\item What about if you pay the loan back in 3 years?
					\vs{1}

				\item What about 4 months?
					\vs{1}

			\end{enumerate}
		\end{ex}
			\newpage
			
		\begin{ex}
			I invest \$500 at $8.5\%$.  How much is the investment worth in 5 years?
		\end{ex}
			\vs{1}

	\subsection*{Discretely Compounding Interest}
		\begin{defn}[Discretely Compounding Interest]
			\textbf{Discretely compounding interest} is interest earned on the balance at discrete time intervals.
		\end{defn}
			\vspace{.1in}

		We have two formulas for discretely compounding interest:
			\begin{center}
				\begin{tabular}{cc}
					$I$			&= \fitb{$\dfrac{r}{n}$}{}\\[15pt]
					$F_d(t)$ 	&= \fitb{$P\lrpar{1+\dfrac{r}{n}}^{nt}$ dollars}{}
				\end{tabular}
			\end{center}
			\vspace{.1in}

		where $P$ is the \showto{ins}{\fbox{principal}}\showto{st}{\blank{1.5}}, $r$ is the \showto{ins}{\fbox{rate (as a decimal)}}\showto{st}{\blank{2}}, $t$ is the \showto{ins}{\fbox{time (in years)}}\showto{st}{\\[15pt]\blank{2}}, and $n$ is the \showto{ins}{\fbox{number of compounds (in a year)}}\showto{st}{\blank{4}}.
			
		\begin{ex}
			You take out a $\$16,750$ loan for a new car.  Find the value of the loan (assuming no payments were made) with:
			\begin{enumerate}[(a)]
				\item $r = 12.5\%$, monthly
					\vs{1}
				\item $r = 6.2\%$, $n = 12$
					\vs{1}
				\item $r = 12.5\%$, yearly
					\vs{1}
				\item $r = 3.79\%$, quarterly
					\vs{1}
					\newpage
					
				\item $r = 3.79\%$, $n = 6$
					\vs{1}
				\item $r = 7.2\%$, daily
					\vs{1}
			\end{enumerate}
		\end{ex}

		\begin{defn}[Annual Percentage Yield]
			The \textbf{annual percentage yield} of an investment (also called the \showto{ins}{\fbox{effective rate}}\showto{st}{\blank{2}}) gives the return on investment in one year.  APY for discretely compounding interest is calculated with the formula
				\fitb{\[APY_{\text{D}} = \left[\lrpar{1+\dfrac{r}{n}}^{n}-1\right]\cdot 100\%\]}{\vspace{1in}}
		\end{defn}

		\begin{ex}
			Calculate the APY for each of the situations from the last example.  Round each to the nearest tenth:
			\begin{enumerate}[(a)]
				\item $r = 12.5\%$, monthly
					\vs{1}

				\item $r = 6.2\%$, $n = 12$
					\vs{1}

				\item $r = 12.5\%$, yearly
					\vs{1}
					\newpage
					
				\item $r = 3.79\%$, quarterly
					\vs{1}

				\item $r = 3.79\%$, $n = 6$
					\vs{1}

				\item $r = 7.2\%$, daily
					\vs{1}

			\end{enumerate}
		\end{ex}
			
		\begin{ex}
			OU Federal Credit Union offers an APR of $6.35\%$ (compounded monthly) for an investment opportunity, while First Fidelity offers you an APY of $5.95\%$.  Which option will give the highest return after one year?
		\end{ex}
			\vs{2}

	\subsection*{Continuously Compounding Interest}
		\begin{defn}[Continuously Compounding Interest]
			Interest earned on the balance at any given time $t$ is called \textbf{continuously compounding interest}, and has the future value formula given by
				\fitb{\[F_c(t) = Pe^{rt}\text{ dollars}\]}{\vspace{1in}}
			where $P$ is the principal, $r$ is the rate, and $t$ is the time.
		\end{defn}
			\newpage
			
			We also have a formula for the APY of continuously compounding interest:
				\fitb{\[APY_{\text{C}} =( e^r-1)\cdot 100\%\]}{\vspace{1in}}

		\begin{ex} Determine the amount that must be invested in the following situations to get $\$7000$ payable in 4 years:
			\begin{enumerate}[(a)]
				\item $3\%$ APR, compounded continuously
					\vs{1}

				\item $3.9\%$ APR, compounded monthly
					\vs{1}

				\item $15.1\%$ APR, simple interest
					\vs{1}

				\item $10\%$ APR, compounded weekly.
					\vs{1}

			\end{enumerate}
		\end{ex}
			\newpage
			
		\begin{ex}
			Find the APY for the examples above, rounding to the nearest hundredth.
			\begin{enumerate}[(a)]
				\item $3\%$ APR, compounded continuously
					\vs{1}

				\item $3.9\%$ APR, compounded monthly
					\vs{1}

				\item $15.1\%$ APR, simple interest
					\vs{1}

				\item $10\%$ APR, compounded weekly.
					\vs{1}
			\end{enumerate}
		\end{ex}
	\clearpage
\end{document}