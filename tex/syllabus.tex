\documentclass[notes]{subfiles}
\begin{document}
\rhead{}
\lhead{}
\chead{}
\thispagestyle{empty}

	\begin{center}
		\textbf{\Large{Section Syllabus}}
	\end{center}
		\vspace{.45in}
		
\begin{center}
	\begin{tabular}{|c|c|c|c|c|c|} \hline
		\textbf{Section} 	& \textbf{Meeting Times} 	& \textbf{Location}	 & \textbf{Instructor} & \textbf{Email} 					& \textbf{Office} \\ \hline
		901 				& TR 4:30pm - 5:45pm 		& Dale Hall 218 & Cory Wilson 		& \texttt{cw@ou.edu}	 	& PHSC 810 \\ \hline
	\end{tabular}
\end{center}

\begin{flushleft}
\textbf{Office Hours}: Office hours will be held in the Math Center (PHSC 232).  I will use Canvas to post these times within the first two weeks of class.  If, for whatever reason, these times are not reasonable for you, send me an email to set up an appointment.  Email correspondence is also an acceptable means by which to ask questions.  Drop-in visits to my office (PHSC 810) are usually fine, but it's a good idea to email me beforehand and set up a meeting.\\ \medskip

\textbf{Course Description}: 
	Topics in differentiation and integration of polynomial, exponential and logarithmic functions. Applications to the business, life and social sciences.
\\ \medskip

\textbf{Course Materials}:
	\emph{Calculus Concepts}, 5th edition, by Latorre et. al.; a TI-83/84 calculator, or a TI-Nspire with an 84 keyboard.  Any calculator with a computer algebra system (CAS) is explicitly \textbf{not allowed}.
\\ \medskip

\textbf{Attendance}:
	Attendance is critical to success in this course (and others).  Students are expected to attend every class; as outlined below, students are responsible for any and all material/information covered or discussed in class.
\\ \medskip

\textbf{Homework}: There is a list of homework problems posted to Canvas.  Students should consider the problems on the list as the \emph{minimum suggested homework} in order to succeed in this class.\\ \medskip

\textbf{Quizzes}: There will be quizzes roughly once per week, each weighted to 10 points. At the end of the semester, the lowest \emph{three} quiz scores will be dropped and the remaining scores will be converted into a 75 point total. \emph{There will be no make up quizzes given, for any reason}. In the event of any extra credit or bonus points available on quizzes, a student's classwork total points cannot exceed 75 points.  Students should expect a quiz on any given day of class.  The purpose of the quizzes is to test comprehension of homework material and conceptual material from any lecture prior to the day on which the quiz is administered- the best way to prepare is to work practice problems, as detailed above.  \\ \medskip

\textbf{Tests}: 
	There will be three exams and a (cumulative) final exam.  The test administration dates are listed on the course calendar below.  Exams I and II will be 75 minutes in length, Exam III will be 90 minutes in length, and the final will be 120 minutes in length.  Students are expected to arrange their schedule to align with this course, \textbf{\emph{including exams}}.  The only acceptable reasons for a makeup will be a normally scheduled class on Tuesday night that runs into the exam time, or a University sponsored school activity in which you directly participate.  All requests for make-up exams MUST be submitted to in writing, using the form I provide, by the last class meeting of the week prior to the exam affected.  Make-up exams will be offered on Wednesday morning, the day after the regularly scheduled exam, in PHSC 108.  In the event of an illness or some unexpected event which prevents you from attending the regularly scheduled exam, it is imperative that you contact me as soon as possible (do NOT wait until you return to class!) and include a way that you can be reached.  
\\ \medskip
	\newpage
	
	\thispagestyle{empty}
\textbf{Classroom Etiquette \& Policies}: 
		Below is a list (not necessarily exhaustive) of specific items which you might consider important:
		\begin{itemize}
			\item All students are expected to come to class prepared, academically or otherwise.  This includes bringing a \emph{physical copy} of the workbook to class every day.  The workbook is posted to our course Canvas page.
			\item Unless authorized by the instructor (or by ADRC accommodation), electronic devices of all forms are explicitly forbidden.  Students in violation of the electronics policy will be warned; repeat offenders will receive a deduction of one point from their overall course grade (each point is worth approximately $0.175\%$).
			\item Students are responsible for \textbf{any and all} material covered or announcements made in class.
			\item Students are expected to participate in the class.  Participation includes (but is not limited to): asking questions of the instructor, asking questions of other students, participating in group activities, etc.
			\item Students are expected to be proactive in communicating any extenuating circumstances with the instructor.  Consideration will be given first to those who communicate their issues to the instructor in a timely manner.  This includes any technical difficulties that the student may face.
		\end{itemize} \medskip
		
\textbf{Math Center}: 
	The Math Center, located in PHSC 232, is a resource which I strongly encourage you to use.  The staffers are able to assist you with whatever you might need; more information is located on the Math Center website, \texttt{www.math.ou.edu/undergrad/mathcenter.html}.
\\ \medskip

\textbf{Academic Misconduct}: 
	You are responsible for reading and abiding by the University's policies concerning academic misconduct. Information about academic misconduct can be found at \texttt{http://integrity.ou.edu}.  From here, you can find a link to the Academic Integrity Code.  \textbf{All} cases of suspected academic misconduct will be referred to the Dean of the College of Arts and Sciences for prosecution under the University's Academic Misconduct Code. In short: don't cheat.  It makes everyone's lives easier.
\\ \medskip
	
\textbf{Accommodation of Disabilities}: 
	Please inform me as soon as possible if you have a disability or special need which requires accommodation in order for you to participate fully in this course. Students with disabilities must be registered with the Accessibility \& Disability Resource Center prior to receiving accommodations in this course. The ADRC is located in the University Community Center, at 730 College Ave, and the website is \texttt{http://www.ou.edu/content/drc.html}. \\ \medskip

\textbf{Grades}: 
	The distribution and grading scale are listed below:
	\begin{center}
		\begin{tabular}{|lc||l||c|} \hline
			Exams 1,2,3 	& 300 points		& 515 - 575 points ($90\% - 100\%$) & \textbf{A}\\
			Final Exam 		& 200 points 		& 458 - 514 points ($80\% - 89\%$)	& \textbf{B}\\
			Quizzes 	 	& 75 points			& 400 - 457 points ($70\% - 79\%$)	& \textbf{C}\\
  							&					& 343 - 399 points ($60\% - 69\%$)	& \textbf{D}\\\hline
			Points Possible	& 575 points		& 0 - 342 points ($< 60\%$)			& \textbf{F}  \\ \hline
		\end{tabular}
	\end{center}
		\newpage
		
		\thispagestyle{empty}
	\begin{center}
		\textsc{\textbf{Tentative Course Schedule}}
	\end{center}
		\emph{The in-class material listed is dynamic, based on speed of the course.  The last day to withdraw with a W is Friday, April 10.}
		
	\begin{center}
		\begin{tabular}{|c||c|c||c|c||c|c||c|}\hline
			\multicolumn{2}{|c|}{\textbf{Tuesday}} & \multicolumn{2}{c|}{\textbf{Thursday}} & \multicolumn{2}{c|}{\textbf{Tuesday}} & \multicolumn{2}{c|}{\textbf{Thursday}}\\ \hline

			\textit{1/14} & Syllabus, 1.1			&\textit{1/16} & 1.1, 1.2		&\textit{3/17}& No Class					&\textit{3/19}& No Class \\ \hline \hline
			\textit{1/21} & 1.2, 1.3				&\textit{1/23} & 1.3, 1.4		&\textit{3/24}& Review, \textbf{Exam 2}	&\textit{3/26}& 3.2\\ \hline \hline
			\textit{1/28} & 1.4, 1.5				&\textit{1/30} & 1.5			&\textit{3/31}& 3.4 						&\textit{4/2}& 3.4, 3.6\\ \hline \hline
			\textit{2/4} 	& 1.8 					&\textit{2/6}& 1.10			&\textit{4/7}	 & 3.6						&\textit{4/9}&  4.2\\ \hline \hline
			\textit{2/11} & Review, \textbf{Exam 1}	&\textit{2/13}& 1.6, 1.7 		&\textit{4/14}& 4.3 						&\textit{4/16}& 4.4	\\ \hline \hline
			\textit{2/18} & 1.9, 1.11, 2.1			&\textit{2/20}& 2.1			&\textit{4/21}& Review, \textbf{Exam 3}	&\textit{4/23}& 4.6 \\ \hline \hline
			\textit{2/25} & 2.2					&\textit{2/27} & 2.3			&\textit{4/28}& Review					&\textit{4/30}& Review\\ \hline \hline
			\textit{3/3}& 2.4						&\textit{3/5}& 2.5			&\multicolumn{4}{c|}{\multirow{2}{*}{\textbf{Final on Wednesday, 5/6}}}\\ \cline{1-4}\cline{1-4}
			\textit{3/10}& 2.6					&\textit{3/12}& 3.1  			&\multicolumn{4}{c|}{}\\ \hline 
		\end{tabular}
	\end{center}
	
	
	Important Dates:
	\begin{itemize}
		\item \textbf{Exam 1}: Tuesday, 2/11/20 (7:30 - 8:45 pm)
		\item \textbf{Exam 2}: Tuesday, 3/24/20 (7:30 - 8:45 pm)
		\item \textbf{Exam 3}: Tuesday, 4/21/20 (7:30 - 9:00 pm)
		\item \textbf{Final}: Wednesday, 5/6/20 (7:30 - 9:30 pm)
		\item \textbf{No class on these days}: Monday - Friday 3/16 - 3/20 (Spring Break)
		\item \textbf{Final day to withdraw with a W}: Friday, April 10.
	\end{itemize}
\end{flushleft}
	\vspace{50pt}
	
	\begin{center}
		\Large{Our exam room for Exams 1, 2, and 3 is}: \makebox[3in]{\hrulefill}
	\end{center}
		\vspace{30pt}
	\begin{center}
		\Large{Our exam room for the Final Exam is}: \makebox[3in]{\hrulefill}
	\end{center}
\end{document}